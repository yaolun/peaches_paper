\documentclass[twocolumn]{aastex62}

\usepackage{amsmath}
\usepackage{graphicx}   % Include figure files
\usepackage{bm}         % bold math
\usepackage{natbib}
\usepackage{etoolbox}
\usepackage{rotating}
\usepackage{array}
\usepackage{booktabs}
\usepackage{amssymb}

% Workaround to only color the year for cite commands
\makeatletter
\patchcmd{\NAT@citex}
  {\@citea\NAT@hyper@{%
     \NAT@nmfmt{\NAT@nm}%
\hyper@natlinkbreak{\NAT@aysep\NAT@spacechar}{\@citeb\@extra@b@citeb}%
     \NAT@date}}
  {\@citea\NAT@nmfmt{\NAT@nm}%
   \NAT@aysep\NAT@spacechar\NAT@hyper@{\NAT@date}}{}{}

% Patch case where name and year are separated by opening bracket
\patchcmd{\NAT@citex}
  {\@citea\NAT@hyper@{%
     \NAT@nmfmt{\NAT@nm}%
\hyper@natlinkbreak{\NAT@spacechar\NAT@@open\if*#1*\else#1\NAT@spacechar\fi}%
       {\@citeb\@extra@b@citeb}%
     \NAT@date}}
  {\@citea\NAT@nmfmt{\NAT@nm}%
\NAT@spacechar\NAT@@open\if*#1*\else#1\NAT@spacechar\fi\NAT@hyper@{\NAT@date}}
  {}{}
\makeatother

% command to write superscript and subscript at the same vertical position
\makeatletter
\DeclareRobustCommand{\textsupsub}[2]{{%
  \m@th\ensuremath{%
    ^{\mbox{\fontsize\sf@size\z@#1}}%
    _{\mbox{\fontsize\sf@size\z@#2}}%
  }%
}}
\makeatother

% units and variables shortcut
\newcommand{\mdot}{\mbox{$\dot M$}}
\newcommand{\lsun}{\mbox{L$_\odot$}}
\newcommand{\msun}{\mbox{M$_\odot$}}
\newcommand{\rsun}{\mbox{R$_\odot$}}
\newcommand{\msunyr}{\mbox{M$_\odot$ yr$^{-1}$}}
\newcommand{\kms}{\mbox{\,km\,s$^{-1}$}}
\newcommand{\mhz}{\mbox{\,MHz}}
\newcommand{\kkms}{\mbox{\,K\,km\,s$^{-1}$}}
\newcommand{\lsmm}{\mbox{$L_{\rm smm}$}}
\newcommand{\lbol}{\mbox{$L_{\rm bol}$}}
\newcommand{\tbol}{\mbox{$T_{\rm bol}$}}
\newcommand{\tb}{\mbox{$T_{\rm b}$}}
\newcommand{\ee}[1]{\mbox{${} \times 10^{#1}$}}% scientific number format
\newcommand{\eten}[1]{\mbox{$10^{#1}$}}% power of ten
\newcommand{\jj}[2]{\mbox{$J = #1\rightarrow#2$}}
\newcommand{\jyb}{\mbox{Jy\,beam$^{-1}$}}
\newcommand{\cc}{\mbox{cm$^{-3}$}}
\newcommand{\h}{\mbox{$^{\text{h}}$}}
\newcommand{\m}{\mbox{$^{\text{m}}$}}
\newcommand{\s}{\mbox{$^{\text{s}}$}}
\newcommand{\dd}{\mbox{$^{\text{d}}$}}
\newcommand{\J}{\mbox{$J$}}
\newcommand{\K}{\mbox{$K$}}
\newcommand{\N}{\mbox{$N$}}
\newcommand{\F}{\mbox{$F$}}
\newcommand{\Ka}{\mbox{$K_\text{a}$}}
\newcommand{\Kc}{\mbox{$K_\text{c}$}}
\newcommand{\vlsr}{\mbox{$v_\text{lsr}$}}
\newcommand{\rt}{\mbox{$\rightarrow$}}
\newcommand{\unc}[2]{\mbox{$^{+#1}_{-#2}$}}

% molecule shortcut
\newcommand{\htcn}{\mbox{H$^{13}$CN}}
\newcommand{\methylformate}{\mbox{CH$_{3}$OCHO}}
\newcommand{\methylformatev}{\mbox{CH$_{3}$OCHO\,$v=1$}}
\newcommand{\methanol}{\mbox{CH$_{3}$OH}}
\newcommand{\tmethanol}{\mbox{$^{13}$CH$_{3}$OH}}
\newcommand{\etmethanol}{\mbox{CH$_{3}^{18}$OH}}
\newcommand{\dmethanol}{\mbox{CH$_{2}$DOH}}
\newcommand{\methanold}{\mbox{CH$_{3}$OD}}
\newcommand{\aminoethanol}{\mbox{NH$_{2}$CH$_{2}$CH$_{2}$OH\,$v_{25}=1$}}
\newcommand{\dimethylether}{\mbox{CH$_{3}$OCH$_{3}$}}
\newcommand{\acetone}{\mbox{CH$_{3}$COCH$_{3}$}}
\newcommand{\ethanol}{\mbox{C$_{2}$H$_{5}$OH}}
\newcommand{\acetaldehyde}{\mbox{CH$_{3}$CHO}}
\newcommand{\ethylcyanide}{\mbox{CH$_{3}$CH$_{2}$CN}}
\newcommand{\methylamine}{\mbox{CH$_{3}$NH$_{2}$}}
\newcommand{\hydroxyacetone}{\mbox{C$_{3}$H$_{6}$O$_{2}$}}
\newcommand{\propane}{\mbox{C$_{3}$H$_{8}$}}
\newcommand{\vinylcyanide}{\mbox{CH$_{2}$CHCN\,$v_{15}=1$}}
\newcommand{\dihydroxyacetone}{\mbox{CH$_{2}$OHCOCH$_{2}$OH}}
\newcommand{\methylcarbamate}{\mbox{NH$_{2}$CO$_{2}$CH$_{3}$\,$v=1$}}
\newcommand{\cyanamide}{\mbox{NH$_{2}$CN}}
\newcommand{\methylcyanideFT}{\mbox{CH$_{3}$C$^{15}$N}}
\newcommand{\methylcyanide}{\mbox{CH$_{3}$CN}}
\newcommand{\dmethylcyanide}{\mbox{CH$_{2}$DCN}}
\newcommand{\hcop}{\mbox{HCO$^{+}$}}
\newcommand{\water}{\mbox{H$_{2}$O}}
\newcommand{\sosigma}{\mbox{SO\,$^{3}\Sigma$}}
\newcommand{\tfso}{\mbox{$^{34}$SO}}
\newcommand{\sotwo}{\mbox{SO$_{2}$}}
\newcommand{\cctht}{\mbox{c-C$_{3}$H$_{2}$}}
\newcommand{\glycolaldehyde}{\mbox{CH$_{2}$OHCHO}}
\newcommand{\formamide}{\mbox{NH$_{2}$CHO}}
\newcommand{\tcooh}{\mbox{$t$-HCOOH}}

% other shortcuts
\newcommand{\refnote}{\textcolor{red}{reference}}


\shorttitle{PEACHES}
\shortauthors{Yang et al.}

\bibliographystyle{aasjournal}
\begin{document}

\title{The Complex Organic Molecules of Embedded Protostars at Perseus}

\author{Yao-Lun Yang}
\affiliation{Department of Astronomy, University of Virginia, Charlottesvile, VA 22904-4235, USA}
\affiliation{RIKEN Cluster for Pioneering Research, Wako-shi, Saitama, 351-0106, Japan}
\author{Nami Sakai}
\affiliation{RIKEN Cluster for Pioneering Research, Wako-shi, Saitama, 351-0106, Japan}
\author{Yichen Zhang}
\affiliation{RIKEN Cluster for Pioneering Research, Wako-shi, Saitama, 351-0106, Japan}

\correspondingauthor{Yao-Lun Yang}
\email{yaolunyang.astro@gmail.com}

% \begin{abstract}
% \end{abstract}
\keywords{}

\section{Introduction}
% The discovery of hot corinos
% - start with hot cores
% - hot corinos: e.g. IRAS 16293, B335, BHR 71, L483
% - long carbon chain molecules: Sakai+, L1527, L483

% Some surveys of the complex chemistry toward protostars
% - Higuchi+2018: The pilot survey that leads to this study
% - Law+, Bergner+, Oberg+
% - Jorgensen+, de la Villarmois+

Planet formation may start during the embedded phase of star formation.  In the scenario where planets form from the embedded disks, resulting in substructures, the chemistry of embedded disks may play a significant role for the chemical composition of the forming planets.  In the recent years, observations discover the emission of carbon-chain molecules and complex organic molecules (COMs) toward the center of several embedded protostars, indicating that embedded protostars have developed a complex chemistry at the disk-forming region.  If the forming planets inherit the chemistry of embedded disks, the abundance of complex organic molecules may implicate future developments of organics on the planets. 

Heavier or more complex molecules, such as cyclic-C$_{3}$H$_{2}$, SO, and complex organic molecules (COMs), are in the gas phase at the inner protostellar envelope ($T\gtrsim100$\,K), exclusively tracing the properties of the inner envelope where a disk may be forming \citep{2013ChRv..113.8961A,2014Natur.507...78S}.  The kinematics of a rotating infalling envelope has been analyzed with the observations of heavier or more complex molecules, such as \methanol\ and \dmethanol\ for HH\,212 \citep{2017ApJ...843...27L}, CS for IRAS\,04365$+$2535 \citep{2016ApJ...820L..34S} and L483 \citep{2017ApJ...837..174O}, cyclic-C$_{3}$H$_{2}$ for L1527 \citep{2014Natur.507...78S}, OCS for IRAS\,16293$-$2422\,A \citep{2016ApJ...824...88O}, and methanol and HCOOH for B335 \citep{2019ApJ...873L..21I}.

In the review by \citet{2009ARA&A..47..427H}, complex molecules are defined as carbon-bearing molecules that contain six atoms or more.  Saturated complex molecules are rich in hydrogen atoms, often called complex organic molecules (COMs), while the unsaturated complex molecules are lack of hydrogen atoms, mostly in the form of long carbon-chain molecules.  While recent observations show several embedded protostars with rich spectra of complex molecules, the occurrence of complex molecules at embedded protostars and its relationship to the star formation process remain poorly understand.  Several protostars are rich in COMs but show little emission of long carbon-chain molecules, such as IRAS 16293$-$2422 \citep{2016A&A...595A.117J}, NGC 1333 IRAS 4A \citep{2004ApJ...615..354B}, B335 \citep{2016ApJ...830L..37I,2019ApJ...873L..21I}, and BHR\,71 (Yang et al. 2020 accepted); some protostars are rich in long carbon-chain molecules but not in COMs, such as L1527 \citep{2010ApJ...722.1633S} and IRAS 15398$-$3359 \citep{2009ApJ...697..769S}.  While the bimodal chemical appearance hints a bimodal evolutionary path, the chemical evolution at the embedded protostars remain ill-constrained as a few protostars show the emission of both COMs and long carbon-chain molecules at different scales, such as L483 \citep{2017ApJ...837..174O}. 

The Perseus ALMA Chemistry Survey (PEACHES) aims to provide the statistics on the occurrence of complex molecules at embedded protostars.  This program unbiasedly observes 51 embedded protostars with ALMA around 260\,GHz, covering the emission of simple molecules such as CS and H$^{13}$CN as well as the emission of complex molecules including \methanol\ and \methylformate.  

\section{Observations}
% Observation details
Set3 has the continuum window ranging from 245800\mhz\ to 246730\mhz, while the Set1 and Set1 have the continuum window ranging from 246200\mhz\ to 247130\mhz.
% imaging parameters

% a map of PEACHES sources

\begin{deluxetable*}{cccccccccc}
    \tabletypesize{\scriptsize}
    \tablecaption{PEACHES Sample \label{tbl:source_list}}
    \tablewidth{\textwidth}
    \tablehead{\colhead{Source} & \colhead{Common names} & \colhead{R.A. (J2000)} & \colhead{Decl. (J2000)} & 
    \colhead{$v_\text{lsr}$} & \colhead{Ref. ($v_\text{lsr}$)} & \colhead{Beam} & \colhead{Cont. Size} & \colhead{$T_\text{cont}$} & \colhead{$M_\text{dust}$\tablenotemark{a}} \\
    \colhead{} & \colhead{} & \colhead{(hh:mm:ss)} & \colhead{(dd:mm:ss)} & 
    \colhead{(km s$^{-1}$)} & \colhead{} & \colhead{(\arcsec)} & \colhead{(\arcsec)} & \colhead{(K)} & \colhead{(M$_{\oplus}$)}}
    \startdata
    Per-emb 22 B   &                & 03:25:22.35    & 30:45:13.11    & 4.3 & S19    & 0\farcs{64}$\times$0\farcs{39} & 0\farcs{95}$\times$0\farcs{51} & 0.92   & 23.1$\pm$13.0                 \\
    Per-emb 22 A   &                & 03:25:22.41    & 30:45:13.26    & 4.3 & S19    & 0\farcs{64}$\times$0\farcs{39} & 0\farcs{86}$\times$0\farcs{65} & 1.71   & 96.4$\pm$24.1                 \\
    L1448 NW       & L1448 IRS 3C   & 03:25:35.67    & 30:45:34.16    & 4.2 & H18    & 0\farcs{64}$\times$0\farcs{39} & 0\farcs{83}$\times$0\farcs{47} & 3.15   & 382.4$\pm$54.9                \\
    Per-emb 33 B/C &                & 03:25:36.32    & 30:45:15.19    & 5.3 & S19    & 0\farcs{64}$\times$0\farcs{39} & 0\farcs{75}$\times$0\farcs{48} & 5.55   & 226.4$\pm$43.7                \\
    Per-emb 33 A   &                & 03:25:36.38    & 30:45:14.72    & 5.3 & S19    & 0\farcs{64}$\times$0\farcs{39} & 0\farcs{73}$\times$0\farcs{45} & 10.33  & 294.0$\pm$57.6                \\
    L1448 IRS 3A   &                & 03:25:36.50    & 30:45:21.90    & 4.6 & H18    & 0\farcs{64}$\times$0\farcs{39} & 0\farcs{85}$\times$0\farcs{59} & 3.21   & 235.1$\pm$47.8                \\
    Per-emb 26     &                & 03:25:38.88    & 30:44:05.28    & 5.4 & S19    & 0\farcs{64}$\times$0\farcs{39} & 0\farcs{69}$\times$0\farcs{45} & 8.03   & 636.1$\pm$102.5               \\
    Per-emb 42     &                & 03:25:39.14    & 30:43:57.90    & 5.8 & S19    & 0\farcs{64}$\times$0\farcs{39} & 0\farcs{64}$\times$0\farcs{39} & 0.66   & 116.5$\pm$29.0                \\
    Per-emb 25     & IRAS 03235$+$3004 & 03:26:37.51    & 30:15:27.81    & 5.5 & S18    & 0\farcs{64}$\times$0\farcs{39} & 0\farcs{69}$\times$0\farcs{41} & 5.27   & 172.6$\pm$65.4                \\
    Per-emb 17     & L1455 IRS 1, IRAS 03245$+$3002 & 03:27:39.11    & 30:13:02.96    & 6.0 & S19    & 0\farcs{64}$\times$0\farcs{40} & 0\farcs{79}$\times$0\farcs{48} & 2.00   & 199.4$\pm$34.0                \\
    Per-emb 20     & L1455 IRS 4    & 03:27:43.28    & 30:12:28.88    & 5.3 & S19    & 0\farcs{64}$\times$0\farcs{40} & 1\farcs{29}$\times$0\farcs{78} & 0.14   & $<$31.1$\pm$15.5              \\
    L1455 IRS 2    &                & 03:27:47.69    & 30:12:04.33    & 5.1 & H18    & 0\farcs{64}$\times$0\farcs{40} & 0\farcs{60}$\times$0\farcs{38} & 0.13   & 23.3$\pm$13.3                 \\
    Per-emb 35 A   & NGC 1333 IRAS 1 & 03:28:37.10    & 31:13:30.77    & 7.4 & Y20    & 0\farcs{66}$\times$0\farcs{42} & 0\farcs{75}$\times$0\farcs{51} & 0.93   & 47.0$\pm$15.7                 \\
    Per-emb 35 B   & NGC 1333 IRAS 1 & 03:28:37.22    & 31:13:31.74    & 7.3 & Y20    & 0\farcs{66}$\times$0\farcs{42} & 0\farcs{78}$\times$0\farcs{53} & 0.75   & 86.8$\pm$20.6                 \\
    Per-emb 27     & NGC 1333 IRAS 2A & 03:28:55.57    & 31:14:36.97    & 6.5 & Y20    & 0\farcs{66}$\times$0\farcs{42} & 0\farcs{93}$\times$0\farcs{66} & 5.79   & 683.7$\pm$93.6                \\
    EDJ2009-172    &                & 03:28:56.65    & 31:18:35.43    & \nodata & \nodata & 0\farcs{66}$\times$0\farcs{42} & 0\farcs{69}$\times$0\farcs{44} & 0.62   & 196.4$\pm$20.0                \\
    Per-emb 36     & NGC 1333 IRAS 2B & 03:28:57.37    & 31:14:15.77    & 6.9 & S19    & 0\farcs{66}$\times$0\farcs{42} & 0\farcs{73}$\times$0\farcs{46} & 5.56   & 666.5$\pm$93.4                \\
    Per-emb 54     & NGC 1333 IRAS 6 & 03:29:01.55    & 31:20:20.49    & 7.9 & S19    & 0\farcs{66}$\times$0\farcs{42} & 0\farcs{69}$\times$0\farcs{40} & 0.07   & 115.5$\pm$28.2                \\
    SVS 13B        & NGC 1333 SVS 13B & 03:29:03.08    & 31:15:51.73    & 8.5 & S19    & 0\farcs{66}$\times$0\farcs{42} & 0\farcs{87}$\times$0\farcs{68} & 6.64   & 581.1$\pm$98.2                \\
    SVS 13A2       & VLA 3          & 03:29:03.39    & 31:16:01.58    & 8.4 & S18    & 0\farcs{66}$\times$0\farcs{42} & 0\farcs{86}$\times$0\farcs{53} & 0.61   & 102.6$\pm$27.7                \\
    Per-emb 44     & NGC 1333 SVS 13A & 03:29:03.76    & 31:16:03.70    & 8.7 & S19    & 0\farcs{66}$\times$0\farcs{42} & 0\farcs{98}$\times$0\farcs{79} & 6.84   & 674.4$\pm$90.9                \\
    Per-emb 15     &                & 03:29:04.06    & 31:14:46.23    & 6.8 & S19    & 0\farcs{66}$\times$0\farcs{42} & 0\farcs{89}$\times$0\farcs{70} & 0.17   & $<$9.3$\pm$6.6                \\
    Per-emb 50     & IRAS 03260$+$3111 A & 03:29:07.77    & 31:21:57.11    & 9.3 & Y20    & 0\farcs{66}$\times$0\farcs{42} & 0\farcs{73}$\times$0\farcs{44} & 4.13   & 535.2$\pm$91.0                \\
    Per-emb 12 B   & NGC 1333 IRAS 4A2 & 03:29:10.44    & 31:13:32.08    & 6.9 & S19    & 0\farcs{66}$\times$0\farcs{42} & 1\farcs{33}$\times$0\farcs{81} & 10.04  & 158.1$\pm$38.4                \\
    Per-emb 12 A   & NGC 1333 IRAS 4A1 & 03:29:10.54    & 31:13:30.93    & 6.9 & S19    & 0\farcs{66}$\times$0\farcs{42} & 1\farcs{11}$\times$0\farcs{98} & 21.85  & 2853.9$\pm$437.0              \\
    Per-emb 21     & NGC 1333 IRAS 7 SM2 & 03:29:10.67    & 31:18:20.16    & 8.6 & Y20    & 0\farcs{66}$\times$0\farcs{42} & 0\farcs{74}$\times$0\farcs{48} & 2.05   & 211.9$\pm$41.1                \\
    Per-emb 18     & NGC 1333 IRAS 7 SM1 & 03:29:11.27    & 31:18:31.09    & 8.1 & S19    & 0\farcs{66}$\times$0\farcs{42} & 0\farcs{84}$\times$0\farcs{73} & 3.42   & 224.8$\pm$47.1                \\
    Per-emb 13     & NGC 1333 IRAS 4B1 & 03:29:12.02    & 31:13:07.99    & 7.1 & S19    & 0\farcs{66}$\times$0\farcs{42} & 1\farcs{07}$\times$0\farcs{83} & 14.76  & 1271.3$\pm$207.6              \\
    IRAS4B'        & NGC 1333 IRAS 4B2 & 03:29:12.85    & 31:13:06.87    & 7.1 & S19    & 0\farcs{66}$\times$0\farcs{42} & 0\farcs{83}$\times$0\farcs{74} & 7.13   & 603.2$\pm$115.2               \\
    Per-emb 14     & NGC 1333 IRAS 4C & 03:29:13.55    & 31:13:58.12    & 7.9 & S19    & 0\farcs{66}$\times$0\farcs{42} & 0\farcs{79}$\times$0\farcs{50} & 3.05   & 311.2$\pm$58.6                \\
    EDJ2009-235    &                & 03:29:18.26    & 31:23:19.73    & 7.7 & Y20    & 0\farcs{67}$\times$0\farcs{42} & 0\farcs{66}$\times$0\farcs{44} & 0.26   & 16.6$\pm$10.0                 \\
    EDJ2009-237    &                & 03:29:18.74    & 31:23:25.24    & \nodata & \nodata & 0\farcs{67}$\times$0\farcs{42} & 0\farcs{67}$\times$0\farcs{42} & 0.12   & \nodata                       \\
    Per-emb 37     &                & 03:29:18.97    & 31:23:14.28    & 7.5 & Y20    & 0\farcs{67}$\times$0\farcs{42} & 0\farcs{82}$\times$0\farcs{57} & 0.56   & 95.4$\pm$23.9                 \\
    Per-emb 60     &                & 03:29:20.05    & 31:24:07.35    & \nodata & \nodata & 0\farcs{67}$\times$0\farcs{42} & 0\farcs{73}$\times$0\farcs{47} & 0.08   & $<$13.3$\pm$8.1               \\
    Per-emb 5      & IRAS 03282$+$3035 & 03:31:20.94    & 30:45:30.24    & 7.3 & S19    & 0\farcs{45}$\times$0\farcs{30} & 0\farcs{56}$\times$0\farcs{41} & 15.29  & 502.3$\pm$86.3                \\
    Per-emb 2      & IRAS 03292$+$3039 & 03:32:17.92    & 30:49:47.81    & 7.0 & S19    & 0\farcs{45}$\times$0\farcs{30} & 1\farcs{35}$\times$0\farcs{97} & 7.41   & 927.4$\pm$175.6               \\
    Per-emb 10     & B1-d           & 03:33:16.43    & 31:06:52.01    & 6.4 & S19    & 0\farcs{46}$\times$0\farcs{30} & 0\farcs{49}$\times$0\farcs{32} & 1.82   & 143.4$\pm$30.7                \\
    Per-emb 40     & B1-a           & 03:33:16.67    & 31:07:54.87    & 7.4 & S19    & 0\farcs{46}$\times$0\farcs{30} & 0\farcs{47}$\times$0\farcs{32} & 1.44   & 72.9$\pm$18.0                 \\
    Per-emb 29     & B1-c           & 03:33:17.88    & 31:09:31.74    & 6.1 & Y20    & 0\farcs{46}$\times$0\farcs{30} & 0\farcs{56}$\times$0\farcs{39} & 8.41   & 233.5$\pm$43.5                \\
    B1-b N         &                & 03:33:21.21    & 31:07:43.63    & 6.6 & C16    & 0\farcs{46}$\times$0\farcs{30} & 0\farcs{56}$\times$0\farcs{47} & 7.67   & 483.8$\pm$85.1                \\
    B1-b S         &                & 03:33:21.36    & 31:07:26.34    & 6.6 & C16    & 0\farcs{46}$\times$0\farcs{30} & 0\farcs{63}$\times$0\farcs{53} & 14.79  & 354.0$\pm$76.5                \\
    Per-emb 16     &                & 03:43:50.97    & 32:03:24.12    & 8.8 & S19    & 0\farcs{50}$\times$0\farcs{32} & 0\farcs{61}$\times$0\farcs{52} & 0.35   & $<$34.9$\pm$18.1              \\
    Per-emb 28     &                & 03:43:51.01    & 32:03:08.02    & 8.6 & S19    & 0\farcs{50}$\times$0\farcs{32} & 0\farcs{56}$\times$0\farcs{32} & 1.52   & 23.5$\pm$16.6                 \\
    Per-emb 1      & HH 211 MMS     & 03:43:56.81    & 32:00:50.16    & 9.4 & S19    & 0\farcs{49}$\times$0\farcs{32} & 0\farcs{68}$\times$0\farcs{48} & 4.57   & 279.6$\pm$50.0                \\
    Per-emb 11 B   & IC 348 MMS     & 03:43:56.88    & 32:03:03.08    & 9.0 & S19    & 0\farcs{50}$\times$0\farcs{33} & 0\farcs{92}$\times$0\farcs{69} & 0.40   & 21.7$\pm$14.9                 \\
    Per-emb 11 A   & IC 348 MMS     & 03:43:57.07    & 32:03:04.76    & 9.0 & S19    & 0\farcs{50}$\times$0\farcs{33} & 0\farcs{61}$\times$0\farcs{48} & 10.47  & 413.5$\pm$73.1                \\
    Per-emb 11 C   & IC 348 MMS     & 03:43:57.70    & 32:03:09.82    & 9.0 & S19    & 0\farcs{50}$\times$0\farcs{33} & 1\farcs{10}$\times$0\farcs{86} & 0.34   & 30.5$\pm$15.4                 \\
    Per-emb 55     & IRAS 03415$+$3152 & 03:44:43.30    & 32:01:31.22    & 12.0 & S19    & 0\farcs{50}$\times$0\farcs{32} & 0\farcs{49}$\times$0\farcs{33} & 0.32   & 56.1$\pm$12.9                 \\
    Per-emb 8      &                & 03:44:43.98    & 32:01:35.19    & 11.0 & S19    & 0\farcs{50}$\times$0\farcs{32} & 0\farcs{49}$\times$0\farcs{36} & 8.51   & 237.9$\pm$47.5                \\
    Per-emb 53     & B5 IRS 1       & 03:47:41.59    & 32:51:43.62    & 10.2 & Y20    & 0\farcs{51}$\times$0\farcs{33} & 0\farcs{58}$\times$0\farcs{42} & 1.55   & 56.6$\pm$22.6                 \\
    \enddata
    \tablerefs{C16$=${\citet{2016A\string&A...586A..44C}}; H18$=${\citet{2018ApJS..236...52H}}; 
               S18$=${\citet{2018ApJS..237...22S}}; S19$=${\citet{2019ApJS..245...21S}}; Y20$=$this study.}
    \tablenotetext{a}{The dust mass is taken from \citet{2020arXiv200602812T}.}
    \tablecomments{The dust masses for L1455 IRS 2, EDJ2009-235, EDJ2009-172 are taken from \citet{2018ApJS..238...19T}
                   by applying a gas-to-dust ratio of 100.  The dust mass of unresolved multiple systems, such as L1448 NW, Per-emb 33 B/C,
                   Per-emb 17, Per-emb 44, Per-emb 27, Per-emb 36, and Per-emb 55, are taken to be the total mass of multiple protostars.
                   The dust mass of Per-emb 40 is assumed to be the dust mass of Per-emb 40 A
                   as the dust mass of Per-emb 40 B is an upper limit.}
\end{deluxetable*}
\input{sources_internal}

\section{Spectra Extraction}
The ALMA image cubes are post-processed to extract 1D spectra for identifying the emission of complex molecules and further analyses.  COMs typically desorb from dust grains at $T \gtrsim 100$\,K, which coincidentally corresponds to $\sim 100$\,au for typical embedded protostars (e.g., \citealt{2020ApJ...891...61Y}).  Given the spatial resolution of $\sim$0\farcs{5} ($\sim$150\,au), we focus on the spectra toward the continuum sources to search for the COMs in the inner envelope.  Four steps of post-processing reduces the image cubes to 1D spectra, which are summarized below.

\begin{itemize}
  \item Continuum fitting: We use the CASA task \texttt{imfit} to iteratively fit for continuum sources down to 5$\sigma$ of the residual image within the central 70\%\ of the primary beam size (20\arcsec).  For the field Set3\_ID09, the fitting uses a threshold of 4$\sigma$ and extends the mask to the entire primary beam as a continuum source is detected toward the edge of the primary beam where the noise is elevated.
  \item Extracting spectra:  We use the CASA task \texttt{specflux} to extract the mean flux density within the ellipse which has the same major and minor axes as well as the position angle as the fitted continuum sources.
  \item Baseline calibration:  The continuum has been removed before the imaging process; however, the extracted spectra sometimes still show imperfect baselines due to rich emission lines, lack of emission, and broad emission features.  Thus, we manually select the frequency ranges for baseline calibration for each spectral window and each field.
  \item Velocity correction:  Finally, the frequency of the extracted spectra are corrected according to the source velocities.  We collect the source velocities from the literature as well as from the strong emission lines in our spectra, such as SO and CS.  Table\,\ref{tbl:source_list} lists the adopted source velocities and the corresponding references.
\end{itemize}

Figure\,\ref{fig:continuum} shows the continuum emission along with the fitted shapes, while the properties of the continuum sources are also listed in Table\,\ref{tbl:source_list}.  Our observations detect 50 continuum sources.    The continuum emission appears as compact circular or elliptical shape with no sub-structure.  Some sources show extended continuum emission resembling the shape of outflow cavities, such as Per-emb 22 A and B.  Three sources, EDJ2009-237, Per-emb 60, and EDJ2009-172, have no spectral line detected and no reliable measurement of source velocity in literature; therefore, we exclude them from spectral extraction as well as the line identification and modeling.  However, these three sources still contribute to the denominator for calculating the detection statistics.

\begin{figure*}[htbp!]
  \centering
  \includegraphics[width=\textwidth]{all_continuum.pdf}
  \caption{The continuum images of all PEACHES protostars.  Non-detections toward L1448 IRS\,2E and NGC 1333 SVS 3 are not shown.  The dashed ellipses illustrate the size of fitted continuum, which is the region for extracting 1D spectra.}
  \label{fig:continuum}
\end{figure*}

\subsection{Line Identification and Modeling}
\label{sec:modeling}
Line identification starts with manual identification and verification for a few sources with rich spectra, including Per-emb-12B and B1-bS.  We use \textsc{splatalogue}\footnote{\href{http://www.splatalogue.net/}{http://www.splatalogue.net/}} to identity the molecular species and use \textsc{xclass} \citep{2017A&A...598A...7M} to verify the identification.  The \textsc{xclass} package is a LTE radiative transfer code that uses the molecular data from the Cologne Database of Molecular Spectroscopy (CDMS; \citealt{2001A&A...370L..49M,2005JMoSt.742..215M,2016JMoSp.327...95E}) and the Jet Propulsion Laboratory (JPL; \citealt{1998JQSRT..60..883P}).  An identification needs to satisfy the following criteria.
\begin{itemize}
  \item The spectra agree with the predicted strengths of the model.
  \item The spectral lines are not all blended with other emission, such as other molecules and the SiO emission tracing the outflows.  The emission of a few species, such as HDCO \&\ \tmethanol, \methanol\ \&\ \methylformate, \acetaldehyde\ \&\ \dmethanol, $^{34}$SO \&\ \ethanol, and \dimethylether\ \&\ \dmethylcyanide, are partially blended (blending occurs at a few lines but other lines remain isolated).  The fittings of those species are performed together to verify their identification.
  \item Identified molecules need to be already found toward young stellar objects as summarized in \citet{2018ApJS..239...17M}.
\end{itemize}
Table\,\ref{tbl:line_id} lists the identified species and transitions that are detected in at least one of the PEACHES protostars.  Only identifiable transitions are listed.  The \textsc{xclass} modeling include all the transitions in our frequency coverage for each molecule regardless their Einstein-A values and upper energy levels.  

Systematic spectral fitting using \textsc{xclass} is applied to all sources using a list of species, compiled from the identifications.  Appendix\,\ref{sec:catalogs} lists the catalogs used in this study.  The fitting function in \textsc{xclass} includes several optimization algorithms that can be used in series to reduce biases.  We configure the algorithm chain that starts with the genetic algorithm followed by the Levenberg-Marquardt $\chi^{2}$ minimization.  The genetic algorithm searches the best-fitting parameters iteratively with generations that evolve like a natural selection, where the better fitting models get less modification over generations.  We setup the genetic algorithm to search for the top two best-fitting models with 30 generations.  Then, the Levenberg-Marquardt $\chi^{2}$ minimization applies to the two best-fitting models for 20 iterations to find the best-fitting models.  The genetic algorithm aims to find local minimums and the Levenberg-Marquardt minimization further finds the best-fitting models in the local minimums.  The two best-fitting models found by the genetic algorithm often very similar, suggesting that there is only one minimum.  To address the rare cases of two separated local minimums, we pick the model with the lower $\chi^{2}$ values from the two best-fitting models constrained by the Levenberg-Marquardt minimization.  

There are five parameters in the \textsc{xclass} modeling, the size of the emitting molecule ($r_\text{COM}$), the excitation temperature ($T_\text{ex}$), the column density ($N_\text{COM}$), the line width ($\Delta \nu$), and velocity offset ($v_\text{off}$).  We assume the COMs are all concentrated at the center, simplified as a 2D thin circular disk.  We fix $r_\text{COM}$ as 0\farcs{5}, similar to our beam size, and optimize the model with five excitation temperatures, 100, 150, 200, 250, and 300 K.  We allow the line width varying between 1.2\kms\ to 3.5\kms\ for better fitting quality but assume no velocity offset from the source velocity, and the allowed range of the column density for each molecule is chosen according to the strength of the emission.  From the fitting results of five $T_\text{ex}$, if a molecule is detected, the mean column densities will be the best-fitting column density, while the range of the column densities indicates the upper and lower uncertainties.  If a molecule is non-detection, the synthetic spectra for all lines are scaled to match the peaks of the each line fitted by a Gaussian profile.  Then, we take the minimum of the corresponding column densities as the upper limit.

% \section{Continuum Opacity}

\section{Detection Statistics}
We summarize the detection statistics in Figure\,\ref{fig:stats}, which includes COMs, the carbon-chain molecules, and the simple organic molecules, such as CS, \htcn, SO, $^{34}$SO, and SO$_{2}$.  In the following results of detection fraction, we include the three sources that are excluded from modeling due to no reliable source velocity, making a total of 50 sources.  The PEACHES protostars show a great chemical diversity from no molecule detected (e.g., B1-bN and L1455 IRS 2) to rich spectra of COMs (e.g. Per-emb 13).  Most of the PEACEHS protostars have simple organics, such as SO, CS, \htcn, and HDCO, and $\sim 60$\%\ of sources have SO$_{2}$ and $^{34}$SO.  Emission of CCH can be easily identified from the spectra.  However, the CCH toward the continuum sources often shows irregular line profiles together with velocity offsets and absorption (Figure\,\ref{fig:all_cch}).  In fact, warm environments, such as the outflow cavity wall, easily enhance the abundance of CCH because of elevated abundance of C$^{+}$.  Thus, CCH emission is extended along with the outflow cavities, making the 1D spectra unrepresentative.

Thirty-four percent of sources have no COMs detected (no emission nor absorption), including Per-emb 8, B1-bN, IRAS 4B$\prime$, Per-emb 36, Per-emb 33 B/C, Per-emb 50, Per-emb 14, Per-emb 28, Per-emb 41, Per-emb 37, Per-emb 11 B, Per-emb 16, Per-emb 55, EDJ2009-235, Per-emb 15, L1455 IRS 2, and Per-emb 54, in the order of decreasing continuum temperatures.  Methanol (\methanol) is detected in 28 sources (56\%); methyl formate (\methylformate) is detected in 15 sources (30\%); and N-bearing COMs are detected in 20 sources (40\%).  Comparing to the COMs in the CALYPSO survey \citep{2020A&A...635A.198B}, the fraction of sources that have methanol, $\sim$50\%, is similar to that for the PEACHES protostars.  Also, 30\%\ of the CALYPSO sources have at least three COMs, while 28\%\ of the PEACHES protostars have at least three COMs.  

The sources with rich spectra COMs tend to have bright continuum emission.  However, the brightest source in continuum, Per-emb 12 A has many molecules detected in absorption due to the high continuum opacity blocking the emission of COMs \citep{2019ApJ...872..196S}.  The number of detected COMs show no obvious correlation with the bolometric luminosity and bolometric temperature of the protostars (Figure\,\ref{fig:stats}), which are conventional evolutionary indicators.  Low luminosity sources have fewer COMs detected; however, if COMs mostly come from thermal desorption, the region with $T > T_\text{desorption}$ may be smaller for the low luminosity sources, making the emission of COMs fainter and reducing our sensitivity to detect COMs.  To test the detection statistics with physical properties of protostars, we collect the mass derived from 9\,mm observations that resolved the sources as a proxy of the central mass \citep{2018ApJS..238...19T}.  The detection statistics with the continuum mass show only tentative correlation with the central mass with smaller central mass have fewer detections of COMs.

Several sources have their SiO emission with a broad line width, significantly contaminating the emission of \ethylcyanide\ and \acetaldehyde.  In the later quantitative discussion, we exclude the spectral windows contaminated by the SiO emission.  For assigning the detections, we can distinguish the emission of \ethylcyanide\ and \acetaldehyde\ from the broad SiO emission in a few sources, such as \acetaldehyde\ in Per-emb 26.

\begin{figure*}[htbp!]
  \includegraphics[width=\textwidth]{stats_sorted_by_Tcont.pdf}
  \caption{The same figure as Figure\,\ref{fig:stats} but sorted by the continuum temperatures listed in Table\,\ref{tbl:source_list}.}
  \label{fig:stats}
\end{figure*}

\renewcommand{\thefigure}{\arabic{figure} (Cont.)}
\addtocounter{figure}{-1}
\begin{figure*}[htbp!]
  \includegraphics[width=\textwidth]{stats_sorted_by_tbol.pdf}
  \caption{The same figure as Figure\,\ref{fig:stats} but sorted by their bolometric temperature.}
\end{figure*}

\addtocounter{figure}{-1}
\begin{figure*}[htbp!]
  \centering
  \includegraphics[width=\textwidth]{stats_sorted_by_lbol.pdf}
  \caption{The detection statistics sorted by their bolometric luminosity.}
\end{figure*}

\addtocounter{figure}{-1}
\begin{figure*}[htbp!]
  \includegraphics[width=\textwidth]{stats_sorted_by_Mcont.pdf}
  \caption{The same figure as Figure\,\ref{fig:stats} but sorted by their mass derived from their 9\,mm observations \citep{2018ApJS..238...19T}.}
\end{figure*}
\renewcommand{\thefigure}{\arabic{figure}}

\section{Correlations of COMs}
The chemical evolution of protostars may leave certain patterns in the abundance of molecules as the dynamical evolution determines the density and temperature structures, which regulate chemical reactions.  Thus, the abundance of COMs and their correlations provide critical information to constrain the chemical evolution at embedded protostars.  The fitted column density of COMs indicates the abundance of COMs around protostars.  Typically COMs are locked into the ices on dust grains at outer envelope. Therefore, we take the column density of COMs as a proxy of the abundance of COMs.  

As described in Section\,\ref{sec:modeling}, we fit the column density and line width with different excitation temperatures, resulting in a range of column density as its uncertainty.  The comparison between CCH and \methanol\ shows no correlation between these two molecules (Figure\,\ref{fig:cch_ch3oh}), similar to the conclusion in \citet{2018ApJS..236...52H}.  The single dish survey by \citet{2016ApJ...833..125G} shows a correlation between C$_{4}$H, a more complex carbon-chain molecules, and \methanol.  Outflow activity can promote the formation of CCH, which is more efficiency at warm temperature.  In face, the morphology of CCH often traces the outflow cavities seen from CS.  Therefore, the lack of correlation between CCH and \methanol\ may be affected by outflows.

Figure\,\ref{fig:corner} shows the correlations of several COMs selected from their detection rates.  The column density of \methanol\ best correlates with that of \methylcyanide.  \citet{2020A&A...635A.198B} also found the tight correlation between these two molecules from the CALYPSO survey, which has a selective sample.  The column densities of \dimethylether\ and \methylformate\ also show a tight correlation.  To quantify the goodness of correlation, we calculate the Pearson's correlation coefficient ($r$), which tests the linearity of two variables.  A simple calculation of the Pearson's correlation coefficient would ignore the uncertainties of the column density.  Thus, we use the bootstrapping method to sample the fitted column densities to calculate Pearson's $r$, by assuming a normal distribution centers on the best-fitted values with the uncertainty as the width of the normal distribution.  If we include the upper limits as normal distributions center on zero, the correlation coefficient becomes significantly lower due to the cluster of samples around zero column density (Figure\,\ref{fig:pearson_distribution}).  With the detection-only sample, the mean Pearson's $r_{d}$ is 0.91, as expected for a tight correlation, with a Gaussian-like distribution skewed toward lower values.  After including the upper limits, the mean Pearson's $r$ decreases to 0.59 with larger uncertainty (the 68\%\ credible interval increases by 160\%).  Thus, the bootstrapped correlation coefficient only considers the detections.

\begin{figure}[htbp!]
  \centering
  \includegraphics[width=0.47\textwidth]{Ncol_ch3oh_ch3cn.pdf}
  \caption{Correlation of the fitted column densities of \methanol\ and \methylcyanide\ from the PEACHES protostars.  The sources where both molecules are detected are shown in black; the sources where only one molecule is detected are shown in magenta; finally, the sources where both molecules are not detected are shown in black for the corresponding upper limits.}
  \label{fig:ch3oh_ch3cn}
\end{figure}

\begin{figure}[htbp!]
  \centering
  \includegraphics[width=0.47\textwidth]{pearson_r_ch3oh_ch3cn.pdf}
  \caption{Distributions of Pearson's correlation coefficient from 10000 resamples drawn from detections $+$ non-detections and only detections.  The legend indicates the mean values of Pearson's $r$ along with the range of the 95\%\ credible interval as the associated uncertainties.}
  \label{fig:pearson_distribution}
\end{figure}

\begin{figure*}[htbp!]
  \centering
  \includegraphics[width=\textwidth]{corner_Ncol_correlations.pdf}
  \caption{Corner plot of the correlations of the column densities between \methanol, \methylcyanide, \methylformate, and \dimethylether.  The color code follows that in Figure\,\ref{fig:ch3oh_ch3cn}.  The dashed line indicates equality.  The legends indicate the Pearson's $r$ for the detection-only sample ($r_{d}$) and the logarithmic ratio of the two molecules (N$_{y}$/N$_{x}$).  The four most detected COMs are shown in this figure, while other COMs are shown in Figure\,\ref{fig:corner_minor}.}
  \label{fig:corner}
\end{figure*}

% \begin{figure*}[htbp!]
%   \centering
%   \includegraphics[width=\textwidth]{corner_Ncol_correlations_norm_Tcont.pdf}
%   \caption{Corner plot of the correlations of the column densities normalized by the continuum brightness temperatures.  The four most detected COMs are shown in this figure, while other COMs are shown in Figure\,\ref{fig:corner_minor}.}
%   \label{fig:corner_tcont}
% \end{figure*}

% \addtocounter{figure}{-1}
% \begin{figure*}[htbp!]
%   \centering
%   \includegraphics[width=\textwidth]{corner_Ncol_correlations_norm_Lbol.pdf}
%   \caption{Corner plot of the correlations of the column densities normalized by the bolometric luminosities.  A few close multiple sources, including Per-emb 12 A \&\ B, Per-emb 35 A \&\ B, and Per-emb 11 A \&\ C, are excluded due to their poorly determined SEDs.}
%   \label{fig:corner_lbol}
% \end{figure*}

% \addtocounter{figure}{-1}
% \begin{figure*}[htbp!]
%   \centering
%   \includegraphics[width=\textwidth]{corner_Ncol_correlations_norm_Tbol.pdf}
%   \caption{Corner plot of the correlations of the column densities are normalized by the bolometric temperatures.  A few close multiple sources, including Per-emb 12 A \&\ B, Per-emb 35 A \&\ B, and Per-emb 11 A \&\ C, are excluded due to their poorly determined SEDs.}
%   \label{fig:corner_tbol}
% \end{figure*}

% \addtocounter{figure}{-1}
% \begin{figure*}[htbp!]
%   \centering
%   \includegraphics[width=\textwidth]{corner_Ncol_correlations_norm_Tcont+Lbol.pdf}
%   \caption{Corner plot of the correlations of the column densities are normalized by the continuum temperature and the bolometric luminosities.  A few close multiple sources, including Per-emb 12 A \&\ B, Per-emb 35 A \&\ B, and Per-emb 11 A \&\ C, are excluded due to their poorly determined SEDs.}
%   \label{fig:corner_tbol}
% \end{figure*}

\begin{figure*}[htbp!]
  \centering
  \includegraphics[width=\textwidth]{corner_Ncol_correlations_combined_norm.pdf}
  \caption{Corner plot of the correlations of the normalized column densities.  The blue, magenta, and black symbols indicate the column densities normalized by the continuum brightness temperature (\tb), while the color scheme follows that in Figure\,\ref{fig:ch3oh_ch3cn}.  The orange symbols show the column densities normalized by \tb\ and the bolometric luminosity ($\lbol$), while the cyan symbols show that normalized by \tb\ and the bolometric temperature (\tbol).  We only show the column densities if both molecules are detected for the ones normalized by \tb\tbol\ and \tb\lbol.  A few close multiple sources, including Per-emb 12 A \&\ B, Per-emb 35 A \&\ B, and Per-emb 11 A \&\ C, are excluded for the normalization of \tbol\ and \lbol\ due to their poorly determined SEDs.}
  \label{fig:corner_combined}
\end{figure*}

\begin{figure*}[htbp!]
  \centering
  \includegraphics[width=\textwidth]{corner_Ncol_correlations_minor.pdf}
  \caption{Corner plot of the correlations of the column densities between \methanol, \methylcyanide, \acetaldehyde, \ethanol, \methylformatev, \acetone, \ethylcyanide, $t$-HCOOH, and \formamide.  The legends are similar to Figure\,\ref{fig:ch3oh_ch3cn}}
  \label{fig:corner_minor}
\end{figure*}

\subsection{Excitation Temperatures}
\subsubsection{\methanol}
The PEACHES spectra cover four methanol lines, while the spectra of each source include three of them due to the frequency shift in the wide spectral window.  The three methanol lines have upper energy ranging from $\sim$50\,K to $\sim$500\,K, which allows us to estimate the rotational temperature of methanol if all three lines are detected.  To construct the methanol rotational diagram, we fit the methanol emission with a Gaussian profile and bootstrap the measurements for fitting the rotational temperature.  Figure\,\ref{fig:rot_dia_example} shows the rotational diagram of Per-emb 22 B along with the sampled rotational temperature.  The derived rotational temperature of methanol ranges from 120\,K to 240\,K with an exception of Per-emb 18, which has a rotational temperature of 395.7\,K for methanol (Table\,\ref{tbl:methanol_rot_temps}).

\begin{figure*}[htbp!]
  \centering
  \includegraphics[width=0.47\textwidth]{Set1_ID03_2.pdf}
  \includegraphics[width=0.47\textwidth]{Set1_ID03_2_rot_temps.pdf}
  \caption{The methanol rotational diagram for Per-emb 22B and the fitted excitation temperature distribution using the bootstraping method.}
  \label{fig:rot_dia_example}
\end{figure*}

\begin{deluxetable}{cc}
  \tabletypesize{\scriptsize}
  \tablecaption{Rotational Temperatures of Methanol \label{tbl:methanol_rot_temps}}
  \tablewidth{0.45\textwidth}
  \tablehead{\colhead{Source} & \colhead{$T_\text{rot}$}}
  \startdata
  Per-emb 26   & 120.1\unc{2.6}{2.5} K   \\
  Per-emb 22 A & 182.4\unc{11.2}{11.4} K \\
  Per-emb 22 B & 157.5\unc{13.4}{13.0} K \\
  Per-emb 17   & 173.9\unc{1.8}{1.9} K   \\
  Per-emb 44   & 197.5\unc{0.3}{0.3} K   \\
  Per-emb 12 B & 194.0\unc{0.8}{0.8} K   \\
  Per-emb 13   & 208.6\unc{3.9}{4.0} K   \\
  Per-emb 27   & 195.8\unc{0.4}{0.4} K   \\
  Per-emb 21   & 151.0\unc{14.6}{15.6} K \\
  Per-emb 35 A & 145.1\unc{3.7}{3.7} K   \\
  Per-emb 18   & 395.7\unc{30.7}{30.4} K \\
  B1-bS        & 241.7\unc{11.7}{11.9} K \\
  Per-emb 29   & 227.7\unc{3.2}{3.3} K   \\
  \enddata
\end{deluxetable}

% \subsubsection{\methylformate}

\section{Spatial Extent of COMs}
\section{Discussion}

\subsection{Chemical Diversity in PEACHES}
\begin{figure*}[htbp!]
  \centering
  \includegraphics[width=0.47\textwidth]{ratio_ch3cn_ch3oh_lbol.pdf}
  \includegraphics[width=0.47\textwidth]{ratio_ch3ocho_ch3oh_lbol.pdf}
  \includegraphics[width=0.47\textwidth]{ratio_ch3och3_ch3oh_lbol.pdf}
  \includegraphics[width=0.47\textwidth]{ratio_ch3och3_ch3ocho_lbol.pdf}
  \caption{Ratios of molecules as a function of \lbol.}
  \label{fig:ratio_lbol}
\end{figure*}

\begin{figure*}[htbp!]
  \centering
  \includegraphics[width=0.47\textwidth]{ratio_ch3cn_ch3oh_tbol.pdf}
  \includegraphics[width=0.47\textwidth]{ratio_ch3ocho_ch3oh_tbol.pdf}
  \includegraphics[width=0.47\textwidth]{ratio_ch3och3_ch3oh_tbol.pdf}
  \includegraphics[width=0.47\textwidth]{ratio_ch3och3_ch3ocho_tbol.pdf}
  \caption{Ratios of molecules as a function of \tbol.}
  \label{fig:ratio_tbol}
\end{figure*}

\begin{figure*}[htbp!]
  \centering
  \includegraphics[width=0.47\textwidth]{ratio_ch3cn_ch3oh_tcont.pdf}
  \includegraphics[width=0.47\textwidth]{ratio_ch3ocho_ch3oh_tcont.pdf}
  \includegraphics[width=0.47\textwidth]{ratio_ch3och3_ch3oh_tcont.pdf}
  \includegraphics[width=0.47\textwidth]{ratio_ch3och3_ch3ocho_tcont.pdf}
  \caption{Ratios of molecules as a function of $T_\text{cont}$.}
  \label{fig:ratio_tcont}
\end{figure*}
\subsection{Comparison to Other Surveys}
\subsection{Complex Chemistry throughout Star Formation}

% \subsection{Source Velocities}
% From the \sosigma\ lines at 258255.8259\,MHz and 261843.7210\,MHz along with the CS \jj{5}{4} line, the averaged source velocities of Per-emb-37 and EDJ2009-172 are 7.437$\pm$0.056\kms\ and 7.748$\pm$0.212\kms, respectively.  The same two \sosigma\ lines also suggest a source velocity of 6.575$\pm$0.060\kms\ for Per-emb-36.

% \begin{table*}
%   \centering
%   \caption{The source velocities of PEACHES sources}
%   \label{tbl:vlsr}
%   \begin{tabular}{ccccc}
%       \toprule
%       Source      & Species   & Frequency [MHz] & Line FWHM [km s$^{-1}$] \\
%       \midrule
%       Set1\_ID01  & SO        & 258255.8        & 5.06$\pm$1.08           \\
%       Set1\_ID01  & HDCO      & 246925.3        & 3.03$\pm$1.58           \\
%       Set2\_ID03  & \methanol & 261805.7        & 3.44$\pm$0.02           \\
%       \bottomrule
%   \end{tabular}
% \end{table*}

\subsection{1D Spectra}

% \begin{deluxetable*}{ccccc}
    \tabletypesize{\scriptsize}
    \tablecaption{Line width fitting \label{tbl:line_width}}
    \tablewidth{\textwidth}
    \tablehead{\colhead{Line} & \colhead{$\nu_\text{rest}$} & \colhead{$\nu_\text{line}$} & 
    \colhead{T$_\text{peak}$} & \colhead{$\Delta\,v$} \\ 
    \colhead{} & \colhead{(MHz)} & \colhead{(MHz)} & \colhead{(K)} & \colhead{(km s$^{-1}$)}}
    \startdata
    \multicolumn{5}{c}{Per-emb-22-B} \\ % Set1_ID03_2
    \hline
    \methanol       & 243915.79 & 243915.39$\pm$0.20 & 1.31$\pm$0.13    & 2.13$\pm$0.25 \\
    SO$_{2}$        & 244254.22 & 244254.06$\pm$0.23 & 1.14$\pm$0.15    & 1.85$\pm$0.29 \\
    \multicolumn{5}{c}{Per-emb-22-A} \\
    \hline
    \multicolumn{5}{c}{L1448NW} \\
    \hline
    \multicolumn{5}{c}{Per-emb-33B/C} \\
    \hline
    \multicolumn{5}{c}{Per-emb-33} \\
    \hline
    \multicolumn{5}{c}{Per-emb-33A} \\
    \hline
    \multicolumn{5}{c}{L1448\,IRS3A} \\
    \hline
    \multicolumn{5}{c}{Per-emb-26} \\
    \hline
    \multicolumn{5}{c}{Per-emb-42} \\
    \hline
    \multicolumn{5}{c}{Per-emb-25} \\
    \hline
    \multicolumn{5}{c}{Per-emb-17} \\
    \hline
    \multicolumn{5}{c}{Per-emb-20} \\
    \hline
    \multicolumn{5}{c}{L1455\,IRS2} \\
    \hline
    \multicolumn{5}{c}{Per-emb-35-A} \\
    \hline
    \multicolumn{5}{c}{Per-emb-35-B} \\
    \hline
    \multicolumn{5}{c}{Per-emb-27} \\
    \hline
    \multicolumn{5}{c}{EDJ2009-172} \\
    \hline
    \multicolumn{5}{c}{Per-emb-36} \\
    \hline
    \multicolumn{5}{c}{Per-emb-54} \\
    \hline
    \multicolumn{5}{c}{SVS13B} \\
    \hline
    \multicolumn{5}{c}{SVS13A2} \\
    \hline
    \multicolumn{5}{c}{Per-emb-44} \\
    \hline
    \multicolumn{5}{c}{Per-emb-15} \\
    \hline
    \multicolumn{5}{c}{Per-emb-50} \\
    \hline
    \multicolumn{5}{c}{Per-emb-12-B} \\
    \hline
    \multicolumn{5}{c}{Per-emb-12-A} \\
    \hline
    \multicolumn{5}{c}{Per-emb-21} \\
    \hline
    \multicolumn{5}{c}{Per-emb-18} \\
    \hline
    \multicolumn{5}{c}{Per-emb-13} \\
    \hline
    \multicolumn{5}{c}{IRAS4B'} \\
    \hline
    \multicolumn{5}{c}{Per-emb-14} \\
    \hline
    \multicolumn{5}{c}{EDJ2009-235} \\
    \hline
    \multicolumn{5}{c}{Per-emb-37} \\
    \hline
    \multicolumn{5}{c}{Per-emb-60} \\
    \hline
    \multicolumn{5}{c}{Per-emb-5} \\
    \hline
    \multicolumn{5}{c}{Per-emb-2} \\
    \hline
    \multicolumn{5}{c}{Per-emb-10} \\
    \hline
    \multicolumn{5}{c}{Per-emb-40} \\
    \hline
    \multicolumn{5}{c}{Per-emb-29} \\
    \hline
    \multicolumn{5}{c}{B1-bN} \\
    \hline
    \multicolumn{5}{c}{B1-bS} \\
    \hline
    \multicolumn{5}{c}{Per-emb-16} \\
    \hline
    \multicolumn{5}{c}{Per-emb-28} \\
    \hline
    \multicolumn{5}{c}{Per-emb-1} \\
    \hline
    \multicolumn{5}{c}{Per-emb-11-B} \\
    \hline
    \multicolumn{5}{c}{Per-emb-11-A} \\
    \hline
    \multicolumn{5}{c}{Per-emb-11-C} \\
    \hline
    \multicolumn{5}{c}{Per-emb-55} \\
    \hline
    \multicolumn{5}{c}{Per-emb-8} \\
    \hline
    \multicolumn{5}{c}{Per-emb-53} \\
    \hline
    \enddata
\end{deluxetable*}

% \section{Analyses}
% Line identification
% The derivation of column density
% - XCLASS fitting
% - the caveats of the XCLASS fitting
%   - the assumption of a Gaussian profile: self absorption, non-Gaussian lines
% - MCMC fitting for methyl formate

% The correlation of column densities

% The morphology of COMs



% \subsection{The Correlations of Column Densities}
% \begin{deluxetable}{cccc}
    \tabletypesize{\scriptsize}
    \tablecaption{MCMC Fitting of Methyl Formate \label{tbl:mf_temp}}
    \tablewidth{\textwidth}
    \tablehead{\colhead{Source} & \colhead{Temperature} & 
               \colhead{log($\mathcal{N}$)} & \colhead{Line width} \\
               \colhead{} & \colhead{[K]} & \colhead{[cm$^{-2}$]} & \colhead{[km s$^{-1}$]}}
    \startdata
Per-emb 33 A & 97.8$^{+126.7}_{-24.6}$ & 15.31$^{+0.20}_{-0.05}$ & 3.7$^{+0.4}_{-0.4}$ \\
Per-emb 26 & 75.0$^{+48.4}_{-9.3}$ & 15.89$^{+0.03}_{-0.06}$ & 4.8$^{+0.2}_{-0.6}$ \\
Per-emb 44 & 213.6$^{+13.6}_{-15.7}$ & 17.21$^{+0.02}_{-0.05}$ & 3.8$^{+0.0}_{-0.1}$ \\
Per-emb 12 B & 199.6$^{+51.4}_{-7.3}$ & 16.93$^{+0.08}_{-0.04}$ & 2.4$^{+0.1}_{-0.0}$ \\
Per-emb 13 & 217.1$^{+43.2}_{-11.9}$ & 16.56$^{+0.08}_{-0.02}$ & 2.1$^{+0.1}_{-0.0}$ \\
Per-emb 27 & 251.3$^{+44.0}_{-16.7}$ & 17.00$^{+0.08}_{-0.05}$ & 4.2$^{+0.1}_{-0.2}$ \\
SVS 13B & 99.5$^{+180.8}_{-25.2}$ & 14.89$^{+0.32}_{-0.06}$ & 2.5$^{+0.5}_{-0.4}$ \\
B1-b S & 120.6$^{+92.9}_{-13.2}$ & 16.08$^{+0.11}_{-0.02}$ & 2.4$^{+0.1}_{-0.1}$ \\
Per-emb 29 & 194.1$^{+58.0}_{-13.2}$ & 16.74$^{+0.08}_{-0.02}$ & 3.5$^{+0.0}_{-0.1}$ \\
Per-emb 11 A & 333.9$^{+56.6}_{-156.0}$ & 16.03$^{+0.09}_{-0.20}$ & 2.4$^{+0.2}_{-0.1}$ \\
    \enddata
\end{deluxetable}

\acknowledgements
Y.-L. Yang acknowledges the supports the JSPS Postdoctoral Fellowship from Japan Society for the Promotion of Science.  This paper makes use of the following ALMA data: ADS/JAO.ALMA\#2016.0.00391.S. ALMA is a partnership of ESO (representing its member states), NSF (USA) and NINS (Japan), together with NRC (Canada), MOST and ASIAA (Taiwan), and KASI (Republic of Korea), in cooperation with the Republic of Chile. The Joint ALMA Observatory is operated by ESO, AUI/NRAO and NAOJ.  The National Radio Astronomy Observatory is a facility of the National Science Foundation operated under cooperative agreement by Associated Universities, Inc.

\facilities{ALMA}

\software{astropy, XCLASS, spectral-cube, CASA}

\appendix
\section{Catalogs for Molecular Data}
\label{sec:catalogs}

\section{Identified Species and Transitions}
Table\,\ref{tbl:line_id} lists the species and their transitions identified from the PEACHES spectra.
\startlongtable
\begin{deluxetable*}{cccccc}
    \tabletypesize{\scriptsize}
    \tablecaption{Line Identification \label{tbl:line_id}}
    \tablewidth{\textwidth}
    \tablehead{\colhead{Frequency (MHz)} & \colhead{Transition\tablenotemark{a}} & 
               \colhead{log(Einstein-A)} & \colhead{$E_\text{u}$ (K)} & \colhead{$g_\text{u}$} & \colhead{Ref.}}
    \startdata
    \multicolumn{6}{c}{Ethynyl (CCH)} \\
    \hline
    262065.00 (0.05) & [3, 5/2, 3]\rt[2, 3/2, 2]\tablenotemark{b}   & $-$4.31 & 25.16  & 7  & CDMS \\
    262067.47 (0.05) & [3, 5/2, 2]\rt[2, 3/2, 1]\tablenotemark{b}   & $-$4.35 & 25.16  & 5  & CDMS \\
    262078.93 (0.02) & [3, 5/2, 2]\rt[2, 3/2, 2]\tablenotemark{b}   & $-$5.22 & 25.16  & 5  & CDMS \\
    \hline
    \multicolumn{6}{c}{Cyclopropenylidene (\cctht)} \\
    \hline
    244222.15 (0.01) & [3, 2, 1]\rt[2, 1, 2]                        & $-$4.23 & 18.17  & 21 & CDMS \\
    246557.77 (0.02) & [16, 10, 7]\rt[16, 9, 8]                     & $-$3.36 & 397.83 & 99 & CDMS \\
    260479.75 (0.02) & [5, 3, 2]\rt[4, 4, 1]                        & $-$3.79 & 44.72  & 33 & CDMS \\
    \hline
    \multicolumn{6}{c}{Methanol (\methanol\ $v_\text{t}=0$)} \\
    \hline
    243915.79 (0.01) & [5, 1, 4]\rt[4, 1, 3] A                      & $-$4.22 & 49.66  & 44 & CDMS \\
    246074.61 (0.02) & [20, 3, 17]\rt[20, 2, 18] A                  & $-$4.08 & 537.03 & 164& CDMS \\
    246873.30 (0.02) & [19, 3, 16]\rt[19, 2, 17] A                  & $-$4.08 & 490.65 & 156& CDMS \\
    261805.68 (0.01) & [2, 1, 1]\rt[1, 0, 1] E                      & $-$4.25 & 28.01  & 20 & CDMS \\
    \hline
    \multicolumn{6}{c}{Methanol (\tmethanol\ $v_\text{t}=0$)} \\
    \hline
    246426.12 (0.22) & [23, 4, 19]\rt[22, 5, 18]                    & $-$4.58 & 721.02 & 47 & CDMS \\
    247086.3 (0.5)   & [23, 3, 20]\rt[23, 2, 21] A$-$\rt\ A$+$      & $-$4.07 & 674.86 & 47 & CDMS \\
    259036.49 (0.17) & [17, 3, 15]\rt[17, 2, 16] A$+$\rt\ A$-$      & $-$4.04 & 396.48 & 35 & CDMS \\
    \hline
    \multicolumn{6}{c}{Methanol (\dmethanol\ $v_\text{t}=0$)} \\
    \hline
    243514.31 (0.01) & [9, 2, 8]\rt[10, 1, 10] o$_\text{1}$         & $-$5.17 & 131.85 & 19 & JPL  \\
    246973.11 (0.01) & [4, 1, 4]\rt[4, 1, 3] e$_\text{1}$           & $-$4.67 & 37.69  & 9  & JPL  \\
    260543.63 (0.01) & [3, 2, 1]\rt[3, 1, 2] o$_\text{1}$           & $-$4.65 & 48.34  & 7  & JPL  \\
    \hline
    \multicolumn{6}{c}{Methanol (\etmethanol\ $v_\text{t}=0$)} \\
    \hline
    246256.60 (0.04) & [11. 2. 10]\rt[10, 3, 7] A                   & $-$4.64 & 184.27 & 92 & CDMS \\
    \hline
    \multicolumn{6}{c}{Sulfur monoxide (\sosigma)} \\
    \hline
    258255.83 (0.01) & [\N, \J]$=$[6, 6]\rt[5, 5]                   & $-$3.67 & 56.50  & 13 & CDMS \\
    261843.72 (0.03) & [\N, \J]$=$[7, 6]\rt[6, 5]                   & $-$3.64 & 47.55  & 15 & CDMS \\
    \hline
    \multicolumn{6}{c}{Sulfur monoxide (\tfso)} \\
    \hline
    246663.47 (0.1)  & [\N, \J]$=$[5, 6]\rt[4, 5]                   & $-$3.74 & 49.89  & 11 & CDMS \\
    \hline
    \multicolumn{6}{c}{Sulfur dioxide (\sotwo)} \\
    \hline
    244254.22 (0.01) & [14, 0, 14]\rt[13, 1, 13]                    & $-$3.79 & 93.90  & 29 & CDMS \\
    \hline
    \multicolumn{6}{c}{Hydrogen cyanide (\htcn)} \\
    \hline
    259010.26 (0.01) & [\J, \F]$=$[3, 3]\rt[2, 3]                   & $-$4.07 & 24.86  & 7  & CDMS \\
    259011.55 (0.01) & [\J, \F]$=$[3, 2]\rt[2, 1]                   & $-$3.19 & 24.86  & 5  & CDMS \\
    259011.80 (0.01) & [\J, \F]$=$[3, 3]\rt[2, 2]                   & $-$3.16 & 24.86  & 7  & CDMS \\
    259011.86 (0.01) & [\J, \F]$=$[3, 4]\rt[2, 3]                   & $-$3.11 & 24.86  & 9  & CDMS \\
    259012.34 (0.01) & [\J, \F]$=$[3, 2]\rt[2, 3]                   & $-$5.46 & 24.86  & 5  & CDMS \\
    259013.89 (0.01) & [\J, \F]$=$[3, 2]\rt[2, 2]                   & $-$3.92 & 24.86  & 5  & CDMS \\
    \hline
    \multicolumn{6}{c}{Carbon Monosulfide (CS)} \\
    \hline
    244935.56 (0.01) & [\J]$=$[5]\rt[4]                             & $-$3.53 & 35.27  & 11 & CDMS \\
    \hline
    \multicolumn{6}{c}{Formaldehyde (HDCO)} \\
    \hline
    246924.6 (0.1)   & [4, 1, 4]\rt[3, 1, 3]                        & $-$3.40 & 37.60  & 9  & CDMS \\
    259034.9 (0.1)   & [4, 2, 2]\rt[3, 2, 1]                        & $-$3.44 & 62.86  & 9  & CDMS \\
    \hline
    \multicolumn{6}{c}{Methyl formate (\methylformate)} \\
    \hline
    245883.2 (0.1)   & [20, 13, 7]\rt[19, 13, 6] E                  & $-$3.89 & 235.98 & 82 & JPL  \\
    245885.2 (0.1)   & [20, 13, 7]\rt[19, 13, 6] A                  & $-$3.89 & 235.98 & 82 & JPL  \\
    245885.2 (0.1)   & [20, 13, 8]\rt[19, 13, 7] A                  & $-$3.89 & 235.98 & 82 & JPL  \\
    245903.7 (0.1)   & [20, 13, 8]\rt[19, 13, 7] E                  & $-$3.89 & 235.97 & 82 & JPL  \\
    246027.5 (0.1)   & [21, 2, 19]\rt[20, 3, 18] E                  & $-$4.63 & 139.85 & 86 & JPL  \\
    246038.9 (0.1)   & [21, 2, 19]\rt[20, 3, 18] A                  & $-$4.63 & 139.85 & 86 & JPL  \\
    246054.8 (0.1)   & [20, 12, 8]\rt[19, 12, 7] E                  & $-$3.84 & 219.43 & 82 & JPL  \\
    246060.8 (0.1)   & [20, 12, 8/9]\rt[19, 12, 7/8] A              & $-$3.84 & 219.43 & 82 & JPL  \\
    246076.9 (0.1)   & [20, 12, 9]\rt[19, 12, 8] E                  & $-$3.84 & 219.41 & 82 & JPL  \\
    246285.4 (0.1)   & [20, 11, 9]\rt[19, 11, 8] E                  & $-$3.80 & 204.21 & 82 & JPL  \\
    246295.1 (0.1)   & [20, 11, 10]\rt[19, 11, 9] A                 & $-$3.80 & 204.21 & 82 & JPL  \\
    246295.1 (0.1)   & [20, 11, 9]\rt[19, 11, 8] A                  & $-$3.80 & 204.21 & 82 & JPL  \\
    246308.3 (0.1)   & [20, 11, 10]\rt[19, 11, 9] E                 & $-$3.80 & 204.20 & 82 & JPL  \\
    246456.1 (0.1)   & [10, 5, 6]\rt[9, 4, 5] E                     & $-$5.52 & 49.09  & 42 & JPL  \\
    246600.0 (0.1)   & [20, 10, 10]\rt[19, 10, 9] E                 & $-$3.77 & 190.34 & 82 & JPL  \\
    246613.4 (0.1)   & [20, 10, 11]\rt[19, 10, 10] A                & $-$3.77 & 190.34 & 82 & JPL  \\
    246613.4 (0.1)   & [20, 10, 10]\rt[19, 10, 9] A                 & $-$3.77 & 190.34 & 82 & JPL  \\
    246623.2 (0.1)   & [20, 10, 11]\rt[19, 10, 10] E                & $-$3.77 & 190.34 & 82 & JPL  \\
    % 246630.0 (0.1)   & [35, 6, 30]\rt[35, 5, 31] A                  & $-$4.77 & 397.98 &142 & JPL  \\
    246660.5 (0.1)   & [10, 5, 6]\rt[9, 4, 5] A                     & $-$4.74 & 49.08  & 42 & JPL  \\
    246675.4 (0.1)   & [15, 4, 12]\rt[14, 3, 11] E                  & $-$4.93 & 81.85  & 62 & JPL  \\
    246683.5 (0.1)   & [15, 4, 12]\rt[14, 3, 11] A                  & $-$4.93 & 81.84  & 62 & JPL  \\
    246752.9 (0.1)   & [10, 5, 5]\rt[9, 4, 5] E                     & $-$4.90 & 49.10  & 42 & JPL  \\
    246891.6 (0.1)   & [19, 4, 15]\rt[18, 4, 14] E                  & $-$3.66 & 126.22 & 78 & JPL  \\
    246914.7 (0.1)   & [19, 4, 15]\rt[18, 4, 14] A                  & $-$3.66 & 126.22 & 78 & JPL  \\
    246945.7 (0.1)   & [10, 5, 6]\rt[9, 4, 6] E                     & $-$4.90 & 49.09  & 42 & JPL  \\
    247040.7 (0.1)   & [20, 9, 11]\rt[19, 9, 10] E                  & $-$3.74 & 177.83 & 82 & JPL  \\
    247044.1 (0.1)   & [21, 3, 19]\rt[20, 3, 18] E                  & $-$3.66 & 139.90 & 86 & JPL  \\
    247053.5 (0.1)   & [21, 3, 19]\rt[20, 3, 18] A                  & $-$3.66 & 139.89 & 86 & JPL  \\
    247057.3 (0.1)   & [20, 9, 12]\rt[19, 9, 11] A                  & $-$3.74 & 177.83 & 82 & JPL  \\
    247057.7 (0.1)   & [20, 9, 11]\rt[19, 9, 10] A                  & $-$3.74 & 177.83 & 82 & JPL  \\
    247063.7 (0.1)   & [20, 9, 12]\rt[19, 9, 11] E                  & $-$3.74 & 177.83 & 82 & JPL  \\
    247124.3 (0.1)   & [10, 5, 5]\rt[9, 4, 6] E                     & $-$4.74 & 49.08  & 42 & JPL  \\
    258275.0 (0.1)   & [21, 13, 8]\rt[20, 13, 7] E                  & $-$3.79 & 248.37 & 86 & JPL  \\
    258277.4 (0.1)   & [21, 13, 8]\rt[20, 13, 7] A                  & $-$3.79 & 248.37 & 86 & JPL  \\
    258277.4 (0.1)   & [21, 13, 9]\rt[20, 13, 8] A                  & $-$3.79 & 248.37 & 86 & JPL  \\
    259341.9 (0.1)   & [24, 0, 24]\rt[23, 1, 23] E                  & $-$4.37 & 158.23 & 98 & JPL  \\
    259342.0 (0.1)   & [24, 1, 24]\rt[23, 1, 23] E                  & $-$3.58 & 158.23 & 98 & JPL  \\
    259342.1 (0.1)   & [24, 0, 24]\rt[23, 0, 23] E                  & $-$3.58 & 158.23 & 98 & JPL  \\
    259342.3 (0.1)   & [24, 1, 24]\rt[23, 0, 23] E                  & $-$4.37 & 158.23 & 98 & JPL  \\
    259342.7 (0.1)   & [24, 0, 24]\rt[23, 1, 23] A                  & $-$4.37 & 158.22 & 98 & JPL  \\
    259342.9 (0.1)   & [24, 1, 24]\rt[23, 1, 23] A                  & $-$3.58 & 158.22 & 98 & JPL  \\
    259343.0 (0.1)   & [24, 0, 24]\rt[23, 0, 23] A                  & $-$3.58 & 158.22 & 98 & JPL  \\
    259343.2 (0.1)   & [24, 1, 24]\rt[23, 0, 23] A                  & $-$4.37 & 158.22 & 98 & JPL  \\
    261822.3 (0.1)   & [17, 10, 7]\rt[17, 9, 8] A                   & $-$4.73 & 156.63 & 70 & JPL  \\
    262088.2 (0.1)   & [16, 10, 6]\rt[16, 9, 7] A                   & $-$4.76 & 146.59 & 66 & JPL  \\
    262088.2 (0.1)   & [16, 10, 7]\rt[16, 9, 8] A                   & $-$4.76 & 146.59 & 66 & JPL  \\
    \hline
    \multicolumn{6}{c}{Methyl formate (\methylformatev)} \\
    \hline
    243511.5 (0.1)   & [20, 12, 8]\rt[19, 12, 7] E                  & $-$3.85 & 407.25 & 82 & JPL  \\
    245846.9 (0.1)   & [21, 3, 19]\rt[20, 3, 18] E                  & $-$3.66 & 326.30 & 86 & JPL  \\
    246106.8 (0.1)   & [20, 7, 14]\rt[19, 7, 13] A                  & $-$3.70 & 343.77 & 82 & JPL  \\
    246184.2 (0.1)   & [20, 8, 13]\rt[19, 8, 12] E                  & $-$3.72 & 353.27 & 82 & JPL  \\
    246187.0 (0.1)   & [21, 2, 19]\rt[20, 2, 18] A                  & $-$3.66 & 326.62 & 86 & JPL  \\
    246233.6 (0.1)   & [20, 7, 13]\rt[19, 7, 12] A                  & $-$3.70 & 343.79 & 82 & JPL  \\
    246274.9 (0.1)   & [20, 7, 13]\rt[19, 7, 12] E                  & $-$3.70 & 343.86 & 82 & JPL  \\
    246410.95 (0.01) & [10, 5, 5]\rt[9, 4, 6] A                     & $-$4.73 & 236.70 & 42 & JPL  \\
    246422.7 (0.1)   & [22, 1, 21]\rt[21, 2, 20] A                  & $-$4.51 & 330.43 & 90 & JPL  \\
    246461.2 (0.1)   & [22, 2, 21]\rt[21, 2, 20] A                  & $-$3.65 & 330.43 & 90 & JPL  \\
    246488.4 (0.1)   & [22, 1, 21]\rt[21, 1, 20] A                  & $-$3.65 & 330.43 & 90 & JPL  \\
    246562.9 (0.1)   & [21, 2, 19]\rt[20, 2, 18] E                  & $-$3.66 & 326.24 & 86 & JPL  \\
    246706.5 (0.1)   & [22, 2, 21]\rt[21, 2, 20] E                  & $-$3.65 & 329.89 & 90 & JPL  \\
    246731.7 (0.1)   & [22, 1, 21]\rt[21, 1, 20] E                  & $-$3.65 & 329.89 & 90 & JPL  \\
    246985.2 (0.1)   & [20, 6, 15]\rt[19, 6, 14] A                  & $-$3.68 & 335.37 & 82 & JPL  \\
    259003.9 (0.1)   & [21, 7, 14]\rt[20, 7, 13] A                  & $-$3.63 & 356.22 & 86 & JPL  \\
    259025.8 (0.1)   & [21, 7, 14]\rt[20, 7, 13] E                  & $-$3.63 & 356.29 & 86 & JPL  \\
    260479.6 (0.1)   & [44, 9, 36]\rt[44, 8, 37] A                  & $-$4.59 & 828.74 & 178& JPL  \\
    \hline
    \multicolumn{6}{c}{Dimethyl ether (\dimethylether)} \\
    \hline
    246499.29 (0.01) & [37, 6, 31]\rt[37, 5, 12] AA                 & $-$4.01 & 693.72 & 750& CDMS \\
    246505.09 (0.01) & [37, 6, 31]\rt[37, 5, 12] AE                 & $-$4.01 & 693.72 & 450& CDMS \\
    246505.09 (0.01) & [37, 6, 31]\rt[37, 5, 12] EA                 & $-$4.01 & 693.72 & 300& CDMS \\
    246697.43 (0.01) & [27, 4, 23]\rt[26, 5, 21] AA                 & $-$4.70 & 367.61 & 330& CDMS \\
    246697.87 (0.01) & [27, 4, 23]\rt[26, 5, 21] EE                 & $-$4.70 & 367.61 & 880& CDMS \\
    246698.31 (0.01) & [27, 4, 23]\rt[26, 5, 21] AE                 & $-$4.70 & 367.61 & 110& CDMS \\
    246698.31 (0.01) & [27, 4, 23]\rt[26, 5, 21] EA                 & $-$4.70 & 367.61 & 220& CDMS \\
    259305.22 (0.01) & [33, 3, 31]\rt[34, 6, 28] AA                 & $-$6.61 & 563.02 & 670& CDMS \\
    259308.39 (0.01) & [33, 3, 31]\rt[34, 6, 28] AE                 & $-$6.61 & 563.02 & 402& CDMS \\
    259308.39 (0.01) & [33, 3, 31]\rt[34, 6, 28] EA                 & $-$6.61 & 563.02 & 268& CDMS \\
    259309.47 (0.01) & [17, 5, 12]\rt[17, 4, 13] AE                 & $-$4.06 & 174.54 & 210& CDMS \\
    259309.76 (0.01) & [17, 5, 12]\rt[17, 4, 13] EA                 & $-$4.06 & 174.54 & 140& CDMS \\
    259311.95 (0.01) & [17, 5, 12]\rt[17, 4, 13] EE                 & $-$4.06 & 174.54 & 560& CDMS \\
    259314.28 (0.01) & [17, 5, 12]\rt[17, 4, 13] AA                 & $-$4.06 & 174.54 & 350& CDMS \\
    \hline
    \multicolumn{6}{c}{Acetone (\acetone)} \\
    \hline
    244218.91 (0.01) & [20, 5, 15]\rt[19, 6, 14] AE                 & $-$3.32 & 139.69 & 82 & JPL \\
    244218.91 (0.01) & [20, 6, 15]\rt[19, 5, 14] AE                 & $-$3.32 & 139.69 & 250& JPL \\
    244218.92 (0.01) & [20, 5, 15]\rt[19, 6, 14] EA                 & $-$3.32 & 139.69 & 160& JPL \\
    244218.92 (0.01) & [20, 6, 15]\rt[19, 5, 14] EA                 & $-$3.32 & 139.69 & 160& JPL \\
    245831.34 (0.09) & [13, 10, 3]\rt[12, 9, 4] EE                  & $-$3.80 & 77.84  & 432& JPL \\ 
    246400.99 (0.05) & [34, 7, 28]\rt[34, 5, 29] EE                 & $-$4.17 & 364.98 &1100& JPL \\
    246400.99 (0.05) & [34, 6, 28]\rt[34, 5, 29] EE                 & $-$4.03 & 364.98 &1100& JPL \\
    246400.99 (0.05) & [34, 7, 28]\rt[34, 6, 29] EE                 & $-$4.03 & 364.98 &1100& JPL \\
    246400.99 (0.05) & [34, 6, 28]\rt[34, 6, 29] EE                 & $-$4.17 & 364.98 &1100& JPL \\
    246404.27 (0.01) & [22, 3, 19]\rt[21, 4, 18] AE                 & $-$3.23 & 149.62 & 90 & JPL \\
    246404.27 (0.01) & [22, 4, 19]\rt[21, 3, 18] AE                 & $-$3.23 & 149.62 & 270& JPL \\
    246404.29 (0.01) & [22, 3, 19]\rt[21, 4, 18] EA                 & $-$3.23 & 149.62 & 180& JPL \\
    246404.29 (0.01) & [22, 4, 19]\rt[21, 3, 18] EA                 & $-$3.23 & 149.62 & 180& JPL \\
    246450.40 (0.01) & [22, 4, 19]\rt[21, 3, 18] EE                 & $-$3.23 & 149.57 & 720& JPL \\
    246450.40 (0.01) & [22, 3, 19]\rt[21, 3, 18] EE                 & $-$5.09 & 149.57 & 720& JPL \\
    246450.40 (0.01) & [22, 3, 19]\rt[21, 4, 18] EE                 & $-$3.24 & 149.57 & 720& JPL \\
    246450.40 (0.01) & [22, 4, 19]\rt[21, 4, 18] EE                 & $-$4.92 & 149.57 & 720& JPL \\
    246496.17 (0.46) & [25, 14, 12]\rt[24, 15, 9] AE                & $-$5.01 & 257.11 & 100& JPL \\
    246496.47 (0.02) & [22, 3, 19]\rt[21, 4, 18] AA                 & $-$3.23 & 149.51 & 270& JPL \\
    246496.47 (0.02) & [22, 4, 19]\rt[21, 3, 18] AA                 & $-$3.23 & 149.51 & 450& JPL \\
    246714.12 (0.05) & [9, 8, 1]\rt[8, 5, 4] EA                     & $-$5.84 & 40.59  & 76 & JPL \\
    246714.94 (0.05) & [32, 4, 28]\rt[32, 4, 29] EA                 & $-$3.97 & 305.61 & 260& JPL \\
    246714.94 (0.05) & [32, 5, 28]\rt[32, 3, 29] EA                 & $-$3.97 & 305.61 & 260& JPL \\
    246715.04 (0.05) & [32, 5, 28]\rt[32, 4, 29] AE                 & $-$3.97 & 305.61 & 390& JPL \\
    246715.04 (0.05) & [32, 4, 28]\rt[32, 3, 29] EA                 & $-$3.97 & 305.61 & 130& JPL \\
    246719.92 (0.04) & [33, 6, 28]\rt[33, 4, 29] EE                 & $-$5.62 & 344.85 &1100& JPL \\
    246719.92 (0.04) & [33, 5, 28]\rt[33, 4, 29] EE                 & $-$3.87 & 344.85 &1100& JPL \\
    246719.92 (0.04) & [33, 6, 28]\rt[33, 5, 29] EE                 & $-$3.87 & 344.85 &1100& JPL \\
    246719.92 (0.04) & [33, 5, 28]\rt[33, 5, 29] EE                 & $-$5.61 & 344.85 &1100& JPL \\
    261818.11 (0.01) & [20, 7, 13]\rt[19, 8, 12] EA                 & $-$3.31 & 151.17 & 160& JPL \\
    261818.17 (0.01) & [20, 7, 13]\rt[19, 8, 12] AE                 & $-$3.31 & 151.17 & 82 & JPL \\
    261819.09 (0.01) & [20, 8, 13]\rt[19, 7, 12] EA                 & $-$3.31 & 151.17 & 160& JPL \\
    261819.17 (0.01) & [20, 8, 13]\rt[19, 7, 12] AE                 & $-$3.31 & 151.17 & 250& JPL \\
    \hline
    \multicolumn{6}{c}{Methyl cyanide (\methylcyanide)} \\
    \hline
    257507.56 (0.01) & [\N, \K]$=$[14, 2]\rt[13, 2]                 & $-$3.00 & 121.28 & 58 & JPL \\
    257522.43 (0.01) & [\N, \K]$=$[14, 1]\rt[13, 1]                 & $-$2.99 & 99.84  & 58 & JPL \\
    257527.38 (0.01) & [\N, \K]$=$[14, 0]\rt[13, 0]                 & $-$2.99 & 92.70  & 58 & JPL \\
    \hline
    \multicolumn{6}{c}{Acetaldehyde (\acetaldehyde\ $v_\text{t}=0$)} \\
    \hline
    246330.73 (0.01) & [15, 3, 13]\rt[15, 2, 14] A                  & $-$4.29 & 131.49 & 62 & JPL \\
    260530.40 (0.01) & [14, 1, 14]\rt[13, 1, 13] E                  & $-$3.20 & 96.39  & 58 & JPL \\
    260544.02 (0.01) & [14, 1, 14]\rt[13, 1, 13] A                  & $-$3.20 & 96.32  & 58 & JPL \\
    260547.46 (2.07) & [9, 4, 5]\rt[9, 3, 7] E, $v_\text{t}=2$      & $-$6.06 & 456.38 & 38 & JPL \\
    \hline
    \multicolumn{6}{c}{gauche-Ethanol ($g$-\ethanol)} \\
    \hline
    246414.76 (0.05) & [14, 3, 11]\rt[13, 3, 10] $v_\text{t}=0$\rt0 & $-$3.89 & 155.72 & 29 & JPL \\
    246524.28 (0.01) & [13, 2, 12]\rt[12, 1, 12] $v_\text{t}=0$\rt1 & $-$4.50 & 136.95 & 27 & JPL \\
    246658.18 (0.01) & [32, 5, 28]\rt[32, 4, 29] $v_\text{t}=0$\rt0 & $-$6.33 & 527.94 & 65 & JPL \\
    246662.98 (0.01) & [4, 2, 3]\rt[3, 1, 3] $v_\text{t}=1$\rt0     & $-$4.36 & 74.77  & 9  & JPL \\
    259322.64 (0.01) & [14, 3, 11]\rt[13, 2, 11] $v_\text{t}=0$\rt1 & $-$4.39 & 155.72 & 29 & JPL \\
    260457.73 (0.01) & [15. 4. 12]\rt[14, 4, 11] $v_\text{t}=1$\rt1 & $-$3.83 & 181.10 & 31 & JPL \\
    \hline
    \multicolumn{6}{c}{trans-Ethanol (\ethanol)} \\
    \hline
    246663.62 (0.05) & [24, 1, 23]\rt[24, 0, 24]                    & $-$3.73 & 252.35 & 49 & JPL \\
    261815.99 (0.05) & [28, 3, 26]\rt[28, 2, 27]                    & $-$3.96 & 350.98 & 57 & JPL \\
    \hline
    \multicolumn{6}{c}{Glycolaldehyde ($cis$-\glycolaldehyde)} \\
    \hline
    246773.09 (0.02) & [30, 2, 28]\rt[30, 1, 29]                    & $-$4.04 & 252.68 & 61 & CDMS \\
    246778.28 (0.02) & [30, 3, 28]\rt[30, 2, 29]                    & $-$4.04 & 252.68 & 61 & CDMS \\
    262056.78 (0.01) & [25, 2, 24]\rt[24, 1, 23]                    & $-$3.34 & 158.25 & 51 & CDMS \\
    261795.48 (0.01) & [25, 11, 14]\rt[25, 10, 15]                  & $-$3.57 & 254.23 & 51 & CDMS \\
    261798.96 (0.01) & [25, 11, 15]\rt[25, 10, 16]                  & $-$3.57 & 254.23 & 51 & CDMS \\
    \hline
    \multicolumn{6}{c}{Methyl cyanide (\dmethylcyanide)} \\
    \hline
    259315.51 (0.01) & [15, 1, 15]\rt[14, 1, 14]                    & $-$2.82 & 104.97 & 31 & CDMS \\
    260523.05 (0.01) & [15, 2, 13]\rt[14, 2, 12]                    & $-$2.82 & 121.60 & 31 & CDMS \\
    \hline
    \multicolumn{6}{c}{Ethyl cyanide (\ethylcyanide)} \\
    \hline
    246268.74 (0.01) & [27, 2, 25]\rt[26, 2, 24]                    & $-$2.90 & 169.80 & 55 & CDMS \\
    246421.92 (0.01) & [28, 2, 27]\rt[27, 2, 26]                    & $-$2.90 & 177.26 & 57 & CDMS \\
    246548.70 (0.01) & [27, 3, 24]\rt[26, 3, 23]                    & $-$2.90 & 174.06 & 55 & CDMS \\
    260535.69 (0.05) & [29, 5, 25]\rt[28, 5, 24]                    & $-$2.84 & 215.06 & 59 & CDMS \\
    \hline
    \multicolumn{6}{c}{Formamide (\formamide)} \\
    \hline
    243521.04 (0.01) & [12, 1, 12]\rt[11, 1, 11]                    & $-$2.98 & 79.19  & 25 & CDMS \\
    \hline
    \multicolumn{6}{c}{Formic acid (\thcooh)} \\
    \hline
    262103.48 (0.01) & [12, 0, 12]\rt[11, 0, 11]                    & $-$3.69 & 82.77  & 25 & CDMS \\
    \enddata
    \tablenotetext{a}{The typical quantum numbers are listed as [\J, \Ka, \Kc] unless specified.}
    \tablenotetext{b}{The quantum numbers are [\N, \J, \F]}
\end{deluxetable*}

\section{The Spectra of CCH}
The CCH spectra toward the continuum emission have irregular line profiles.  Some spectra have strong self-absorption, while some spectra only show the blue-shifted emission.  Due to the absorption and irregular line profile, the \textsc{xclass} fitting routine often fails to faithfully reproduce the observed CCH spectra.  CCH can easily form in the outflow cavity wall due to the abundant CH$_{4}$ sublimated from dust grains as well as C$^+$ ionized by the UV radiation.  Thus, the CCH spectra can have broad line width and multiple components.  Furthermore, the morphology of the CCH emission traces the outflows, making our extraction from the continuum emission non-ideal for representing the nature of the CCH emission.  Figure\,\ref{fig:all_cch} and Y show the spectra and the moment 0 map of CCH, respectively.

\begin{figure}[htbp!]
  \centering
  \includegraphics[width=0.47\textwidth]{Ncol_ch3oh_cch.pdf}
  \caption{Correlation of the column densities of CCH and \methanol\ fitted from the PEACHES protostars.  The sources where both molecules are detected are shown in black; the sources where only one molecule is detected are shown in magenta; finally, the sources where both molecules are not detected are shown in black for the corresponding upper limits.}
  \label{fig:cch_ch3oh}
\end{figure}

\begin{figure*}[htbp!]
  \centering
  \includegraphics[width=\textwidth]{all_cch.pdf}
  \caption{The CCH spectra of all PEACHES sources extracted from the continuum emission.}
  \label{fig:all_cch}
\end{figure*}

\section{Notes on the COMs Modeling}
\section{Notes on the 1D Spectra}
% \paragraph{L1448 IRS 3A}
% \begin{itemize}
%   \item The emission of SO and CS shows additional red-shifted component, separated by $\sim$4.6\kms.
% \end{itemize}

\paragraph{Per-emb-33-A}
\begin{itemize}
    \item The fitting of \methylformate\ reproduces the the strongest emission at 259343\mhz, but underestimates the emission between 246275\mhz\ to 247070\mhz, where the emission is at most 0.5\,K.  Considering the narrow absorption in HCN, CS, and SO lines as well as the brighter continuum temperature (10.5\,K), the emission of \methylformate\ may be affected by the continuum opacity.
%   \item Strong absorption at 246509\mhz.
%   \item HDCO seems to have two components for the transition at 246925\mhz, but the one at 259035\mhz\ only has one component, which is underestimated by the model.
\end{itemize}

% \paragraph{Per-emb-33-B/C}
% \begin{itemize}
%   \item CCH shows three components leading to an inaccurate fit.
% \end{itemize}

% \paragraph{Per-emb-42}
% \begin{itemize}
%   \item Double-peaked CS line
%   \item Triple-peaked CCH line
% \end{itemize}

\paragraph{Per-emb-26}
\begin{itemize}
  % \item Red-shifted excess appears in the CS, SO, \methanol, \htcn, and HDCO lines.
  \item Red-shifted excess appears in the \methanol\ lines.
  % \item Broad SO lines peak at slightly blue-shifted velocity ($\sim$1\kms).
  % \item The best-fitting model overestimates the \ethylcyanide, \acetaldehyde, and \cctht\ lines, possibly due to the contamination of SiO, which should have been excluded for fitting.  Tests are running now.
  % \item The secondary \methanol\ line becomes overestimated at $T_\text{ex} \geq $200 K.
  \item Unidentified lines at 246525\mhz\ and 244249\mhz.
\end{itemize}

% \paragraph{Per-emb-22-B}
% \begin{itemize}
%   \item The SO lines have small red-shifted excess.  The best-fitting line strength decreases significantly at $T_\text{ex} = $200 K.  Not sure why.
%   \item The SO line profiles are slightly skewed toward blue-shifted velocity, while the CS line shows another brighter peak at blue-shifted velocity.
% \end{itemize}

% \paragraph{Per-emb-22-A}
% \begin{itemize}
%   \item Hints of \methylformate emission, but not significant enough to warrant a detection.
% \end{itemize}

\paragraph{Per-emb-17}
\begin{itemize}
  \item Many line profiles exhibit a broad double-peaked profile, separated by $\sim$5--6\kms.  Per-emb-17 is a binary system unresolved by our observations.  However, the channel maps suggest that the two components are likely to surrounding the southern source, Per-emb-17-B.
  \item The \methylformate\ line at $\sim$259343\mhz\ may be optically thick.
\end{itemize}

% \paragraph{Per-emb-20}
% \begin{itemize}
%   \item The CCH lines have a narrow double-peaked profile.  The CS line shows a similar double-peak profile.
%   \item The SO lines have a broad component underneath the typical narrow lines.
% \end{itemize}

\paragraph{SVS13 A2}
\begin{itemize}
  \item Weak indication of the unidentified line at 246525\mhz, which has been detected in other sources.
  % \item The model strength of CCH suddenly decreases at $T_\text{ex} = $200 K by $\sim$0.5 K over a 2 K line.
\end{itemize}

\paragraph{Per-emb-44}
\begin{itemize}
  \item Unidentified lines at 244248\mhz, 246219\mhz, 246254\mhz, 246344\mhz, 246389\mhz, 246434\mhz, 246525\mhz, 246838\mhz, 258268\mhz, 258271\mhz, and 262068-262070\mhz.
  \item Higher temperatures ($T\text{ex} > $100 K) provide better fittings.  Probably should adopt the temperature fitted from \methylformate\ (previous MCMC fitting suggests a temperature of 263 K).
\end{itemize}

\paragraph{Per-emb-12-B}
\begin{itemize}
  \item Unidentified lines at 244248\mhz, 246254\mhz, 246314\mhz, 246322\mhz, 246389\mhz, 246434\mhz, 246525\mhz, 246696\mhz, 246838\mhz, 246873\mhz, 247082\mhz, 258268\mhz, 258271\mhz, and 262068-262070\mhz.
\end{itemize}

\paragraph{Per-emb-12-A}
\begin{itemize}
  \item Strong absorption features detected across the spectra, CCH, SO, \htcn, CS, \methanol, HDCO, \methylcyanide, and \methylformate.
\end{itemize}

\paragraph{IRAS4B$\prime$}
\begin{itemize}
  \item Spectra show no emission along with absorption at SO, CS, and \methanol\ lines.
\end{itemize}

\paragraph{Per-emb-13}
\begin{itemize}
  \item The \methylformate\ emission needs $T_\text{ex} > $100 K to have a good fit.
  \item All three \methanol\ lines are detected but two of them show clear sign of self-absorption, therefore, not ideal for fitting the excitation temperature.
  \item Unidentified lines at 244248\mhz, 246254\mhz, 246331\mhz, 246344\mhz, 246434\mhz, 246525\mhz, 246838\mhz, 246974\mhz, 247086\mhz, 257268\mhz, 257271\mhz, 259323\mhz, 259331\mhz, 262098\mhz, and 262109\mhz.
  % \item The best-fitting models have two different widths for the \acetaldehyde\ lines.  
  \item The best-fitting model for \tmethanol\ lines overestimates the line width due to the weak and broad line at 247086\mhz.
\end{itemize}

\paragraph{Per-emb-27}
\begin{itemize}
  \item All three \methanol\ lines are detected, but none of the temperature produce a good fit to all three lines, suggesting that some lines are optically thick.  The intensities of the transitions at 243916\mhz\ and 261806\mhz\ are $\sim$30 K, while the intensity at 246873\mhz\ is about 24 K.  They seems to be optically thick.  In comparison, the continuum brightness temperature is only 5.8 K.
  \item Unidentified lines at 244232\mhz, 244248\mhz, 246207\mhz, 246254\mhz, 246388\mhz, 246435\mhz, 246525\mhz, 246538\mhz, 246838\mhz, 246973\mhz, 247084\mhz, and 259330\mhz.
  \item The \methanol\ line at 243916\mhz\ and the SO lines become optically thick at 100 K.
\end{itemize}

% \paragraph{Per-emb-54}
% \begin{itemize}
%   \item The SO lines appears red-shifted by $\sim$2\kms.
%   \item The emission of CS and CCH shows a blue-shifted peak along with absorption slightly red-shifted compared to the source velocity.
% \end{itemize}

\paragraph{Per-emb-21}
\begin{itemize}
  \item Emission of \methanol\ is detected.  However, the broad width and noisy spectra lead to a bad fit.  The best-fitting model has the maximum line width allowed, 3.5\kms.
\end{itemize}

% \paragraph{Per-emb-14}
% \begin{itemize}
%   \item The best-fitting model underestimates the HDCO lines, possibly due to the simultaneously fitted \tmethanol\ lines.
% \end{itemize}

\paragraph{Per-emb-35-B}
\begin{itemize}
  % \item The best-fitting model underestimates the SO lines at $T_\text{ex} = $100 K.
  % \item The CCH lines show a double-peaked line profile.
  \item The \methanol\ line at 243915\mhz has an S/N of 1.2, but hints the existence of \methanol.
\end{itemize}

\paragraph{Per-emb-35-A}
\begin{itemize}
  \item The goodness of fitting for the \methanol\ lines is a strong function of temperature, suggesting that the \methanol\ lines can indicate the $T_\text{ex}$.
  \item The \methylformate\ line at 259342\mhz\ has an S/N of 1.8, but hint the existence of \methylformate.
\end{itemize}

% \paragraph{SVS13B}
% \begin{itemize}
%   \item The fitted width of the HDCO lines is overestimated.
% \end{itemize}

\paragraph{Per-emb-15}
\begin{itemize}
  \item All lines have only the blue-shifted emission, making them blue-asymmetric.
\end{itemize}

% \paragraph{Per-emb-50}
% \begin{itemize}
%   \item The SO and SO$_{2}$ lines have a very broad component ($\Delta \nu = $6\kms), skewing toward the blue-shifted velocity.
% \end{itemize}

\paragraph{Per-emb-18}
\begin{itemize}
  \item Many transitions of \methylformate\ are tentatively detected; however, none of them has S/N $>$ 3.  Currently categorized as non-detection.
  % \item The HDCO lines are broader than the maximum allowed line width, 3.5\kms, making the fitting inaccurate.
\end{itemize}

% \paragraph{Per-emb-37}
% \begin{itemize}
%   \item The fitting of SO lines has a strong variation as a function of temperatures.
% \end{itemize}

% \paragraph{Per-emb-36}
% \begin{itemize}
%   \item The SO lines are underestimated by the best-fitting model.  Perhaps it can be fixed by changing the \texttt{Variation} parameter.  To be tested.  Test run on laptop shows no issue of having a good fit at 100 K for SO with \texttt{Variation} $= 10^{-2}$.
%   \item The CCH lines only show at the blue-shifted velocities.
% \end{itemize}

\paragraph{B1-bS}
\begin{itemize}
  \item Higher temperatures produce worse fittings to the \methylformate\ lines.  Previous MCMC fitting of the \methylformate\ lines suggests a temperature of 58 K.
  \item The fitting of \dimethylether\ is limited by the minimum line width of 1.2\kms.  
  \item Unidentified lines at 246027\mhz, 246099\mhz, 246143\mhz, 246192\mhz, 246525\mhz, 246674\mhz, and 2467320\mhz.
  \item The \methylformate\ lines around 258278\mhz\ and the \htcn\ lines have a few dips within the line profile, suggesting absorption or just noisy spectra.
\end{itemize}

\paragraph{Per-emb-29}
\begin{itemize}
  \item Only two \methanol\ lines are covered.  Both lines have a strength of $\sim$10 K, suggesting optically thick.
\end{itemize}

% \paragraph{Per-emb-5}
% \begin{itemize}
%   \item The CCH fitting is not robust across different temperatures (only 150 K and 300 K fit).
% \end{itemize}

\bibliography{research,fixed}

\end{document}
